% CREATED BY DAVID FRISK, 2016

% IMPORT SETTINGS
\documentclass[12pt,a4paper]{report}
\input{settings}


\begin{document}
% COVER PAGE, TITLE PAGE AND IMPRINT PAGE
\pagenumbering{roman}			% Roman numbering (starting with i (one)) until first main chapter
% CREATED BY DAVID FRISK, 2016
% MODIFIED BY ALEXANDER HÅKANSSON, 2017

% COVER PAGE
\begin{titlepage}
\newgeometry{top=3cm, bottom=3cm,
			left=2.25 cm, right=2.25cm}	% Temporarily change margins		
			
% Cover page background 
\AddToShipoutPicture*{\backgroundpic{-4}{56.7}{figure/front/frontpage-en-GU.pdf}}
\addtolength{\voffset}{2cm}

% Cover picture (replace with your own or delete)		
\begin{figure}[H]
\centering
\vspace{2cm}	% Adjust vertical spacing here
\includegraphics[width=0.9\linewidth]{figure/Wind.png}
\end{figure}

% Cover text
\mbox{}
\vfill
\renewcommand{\familydefault}{\sfdefault} \normalfont % Set cover page font
\textbf{{\Huge 	A Chalmers University of Technology 	\\[0.2cm] 
				Bachelor's thesis template for \LaTeX}} 	\\[0.5cm]
{\LARGE A Subtitle that can be Very Much Longer}\\[0.2cm]
Bachelor of Science Thesis in Computer Science and Engineering \setlength{\parskip}{0.5cm}

{\Large JOHN DOE} \setlength{\parskip}{1.9cm}

Chalmers University of Technology \\
University of Gothenburg \\
Department of Computer Science and Engineering \\
Göteborg, Sweden, June 2017


%\renewcommand{\familydefault}{\rmdefault} \normalfont % Reset standard font
\end{titlepage}

\begin{comment} % Remove comment to get blank page
% BACK OF COVER PAGE (BLANK PAGE)
\newpage
\restoregeometry
\thispagestyle{empty}
\mbox{}
\end{comment}

% TITLE PAGE
\newpage
\setcounter{page}{1}
\thispagestyle{empty}
\begin{center}
	\large Bachelor of Science Thesis\\[4cm]		% Report number given by department 
	\textbf{\large An Informative Headline describing the Content of the Report} \\[0.7cm]
	{\large A Subtitle that can be Very Much Longer if Necessary}\\[1cm]
	\textsc{{\large Some Author}}\\
	\textsc{{\large Other Author}}\
	
	\vfill	
	% Logotype on titlepage	
	\begin{figure}[H]
	\centering
	% Remove the following line to remove the titlepage logotype
	%\includegraphics[width=0.2\pdfpagewidth]{figure/front/logo_eng.pdf} \\	
	\end{figure}	\vspace{5mm}	
	
	Department of Computer Science and Engineering \\
	\textsc{Chalmers University of Technology} \\
	University of Gothenburg \\[0.5cm]
	Göteborg, Sweden 2017 \\
\end{center}


% IMPRINT PAGE (BACK OF TITLE PAGE)
\newpage
\thispagestyle{plain}
%\vspace*{4.5cm}
\textbf{An Informative Headline describing the Content of the Report}\\
A Subtitle that can be Very Much Longer if Necessary\\
SOME AUTHOR\\
OTHER AUTHOR\setlength{\parskip}{0.7cm}

\copyright ~ SOME AUTHOR, OTHER AUTHOR, 2017 \setlength{\parskip}{0.7cm}

Examiner: Name, Department \setlength{\parskip}{1cm}

Department of Computer Science and Engineering\\
Chalmers University of Technology\\
University of Gotehnburg\\
SE-412 96 Göteborg\\
Sweden\\
Telephone: +46 (0)31 772 1000 \setlength{\parskip}{0.5cm}

\vfill
The Author grants to Chalmers University of Technology and University of Gothenburg the non-exclusive right to publish the Work electronically and in a non-commercial purpose make it accessible on the Internet.\\
The Author warrants that he/she is the author to the Work, and warrants that the Work does not contain text, pictures or other material that violates copyright law.\\\\
The Author shall, when transferring the rights of the Work to a third party (for example a publisher or a company), acknowledge the third party about this agreement. If the Author has signed a copyright agreement with a third party regarding the Work, the Author warrants hereby that he/she has obtained any necessary permission from this third party to let Chalmers University of Technology and University of Gothenburg  store the Work electronically and make it accessible on the Internet.


\vfill
% Caption for cover page figure if used, possibly with reference to further information in the report
Cover:\\
Wind visualization constructed in Matlab showing a surface of constant wind speed along with streamlines of the flow. \setlength{\parskip}{0.5cm}

Department of Computer Science and Engineering\\
Göteborg, Sweden 2017


 

% ABSTRACT
\newpage
% CREATED BY DAVID FRISK, 2016
% MODIFIED BY ALEXANDER HÅKANSSON, 2017
\large
\textbf{\varthetitle}\\
\varthesubtitle\\[0.7cm]
JESPER JAXING\\
ALEXANDER HÅKANSSON\\
MAXIM GORETSKYY\\
GMAL TCHAEFA\\
AXEL OLIVECRONA\\
JONATAN ALMÉN\\
\normalsize
\textit{Department of Computer Science and Engineering\\
Chalmers University of Technology\\
University of Gothenburg}\\[0.7cm]
Bachelor of Science Thesis
\setlength{\parskip}{0.5cm}

\thispagestyle{plain}			% Supress header 
\setlength{\parskip}{0pt plus 1.0pt}
\section*{Abstract}
Lorem ipsum dolor sit amet, consectetur adipisicing elit, sed do eiusmod tempor incididunt ut labore et dolore magna aliqua. Ut enim ad minim veniam, quis nostrud exercitation ullamco laboris nisi ut aliquip ex ea commodo consequat. Duis aute irure dolor in reprehenderit in voluptate velit esse cillum dolore eu fugiat nulla pariatur. Excepteur sint occaecat cupidatat non proident, sunt in culpa qui officia deserunt mollit anim id est laborum.

% KEYWORDS (MAXIMUM 10 WORDS)
\vfill
\textbf{Keywords:} lorem, ipsum, dolor, sit, amet, consectetur, adipisicing, elit, sed, do.

\newpage				% Create empty back of side
\thispagestyle{empty}
\mbox{}

% ACKNOWLEDGEMENTS
\newpage
% CREATED BY DAVID FRISK, 2016
\thispagestyle{plain}			% Supress header
\section*{Acknowledgements}
We would like to thank our supervisor Olof Mogren for the great insight and feedback that he has provided during the work with project, making it a fun process and enabling us to succeed. We would also like to thank the division for language and communication for their tremendous help with producing the writing of this thesis. Furthermore, the feedback we have received from other bachelors thesis groups have been of great help.

\newpage				% Create empty back of side
\thispagestyle{empty}
\mbox{}


% TABLE OF CONTENTS
\newpage
\tableofcontents

% OTHER FRONTMATTER
% List of figures (add to table of contents)
\cleardoublepage
\addcontentsline{toc}{chapter}{\listfigurename} 
\listoffigures
% List of tables (add to table of contents)
\cleardoublepage
\addcontentsline{toc}{chapter}{\listtablename}  
\listoftables


% START OF MAIN DOCUMENT
\cleardoublepage
\setcounter{page}{1}
\pagenumbering{arabic}			% Arabic numbering starting from 1 (one)
\setlength{\parskip}{0pt plus 1pt}

\clearpage



% INTRODUCTION
\chapter{Introduction}

This thesis will evaluate if it is possible to minimise the irrelevant notifications by using an ANN. In today’s digital age it can be hard keeping track of all the content and conversations on different social platforms; such as Slack, Facebook Messenger, WhatsApp and Reddit. People do not want to miss the conversations or news that might interest them, rather get instant feedback when these conversations or news take place. However, the notifications of today are indiscriminate in that it is hard to get notified when something of interest is posted while not being notified at all times. Some applications have tried to solve this by allowing manual “mentioning” or “highlighting” of a person to get their attention; Slack is one example of this \parencite{slack}. However, these functions put a constraint on the users where they need to figure out by themselves who might be interested in some content. It is also deemed as bad manners to highlight a person - or a group of people - too often. Studies have even shown that an excessive amount of notifications can lead to stress \parencite{relaffinity}.
\\\\
One solution to this problem could be a software that when given new content would automatically highlight/notify specific users that is deemed interested. Such a software would relieve a user of the stress of an abundance of notifications and contribute to an overall better experience for them. Even if it would not be perfect, a less amount of irrelevant notifications could potentially make a big difference.
%Such a middle ground would be beneficial for both the users and the owner of the platform. It would make it easier for users to find interesting content which would improve the overall experience for the average user - this, in turn, benefits the owner of the platform as its users get more satisfied overall.
\\\\
Similar problems to the above mentioned have been solved in other fields like electronic commerce. Amazon uses a technique called recommender systems to recommend new items for their users \parencite{amazonfiltering}. Their recommender system is not based on ANN but rather compares the similarity between new products and products that a user has previously bought or rated \parencite{amazonfiltering}. What motivates the use of ANN over recommender systems in this project is their ability to capture a deeper meaning of text content which we believe could be useful when modelling what people like or dislike. 

\section{Purpose}
The aim of this project is to evaluate the performance of deep recurrent neural networks to personalise recommendations for users on social platforms. By using a neural network trained for this task an application should then be able to use its predictions to individually notify users when content interesting to them is published.

\section{Problem Description}
To complete the purpose of the project, the correlation between some content on the platform and a users' interest needs to be modelled. This will be attempted using an artificial neural network. In order to successfully model this problem there are a number of factors that must be considered.
\\\\
First of all, in order to use an artificial neural network a dataset which can be used for training is needed. When a dataset is obtained, the ANN must be modelled to capture the relation between the contents of a text and a user's interest. The modelling is a key part of the problem where a lot of experimentation with hyperparameters is needed. The number of layers, their sizes and different functions in the network needs to be examined. Once an optimal performance has been achieved it also needs to be compared with some baseline in order to evaluate whether this method is good in practice or not. These problems, as described in more detail below,  must be addressed and solved in order to achieve the goal of the project.

\subsection{Selecting a Dataset}\label{sec:select_dataset}
For all datasets available not all are suited to the task. The dataset needs to satisfy the following constraints: 
\vspace*{0.25cm}
\begin{itemize}
    \item It must be sufficiently large
    \item It must be labelled
\end{itemize}
\vspace*{0.25cm}
As part of this project, potential datasets that fulfilled the above constraints were examined to select the most suitable one.

\subsection{Finding Hyperparameters}
The hyperparameters for the ANN will in the end determine its performance. This is since the hyperparameters determine characteristics of the ANN - its shape, size and behaviour. Some of the hyperparameters that were  examined are described below:
\vspace*{0.25cm}
\begin{itemize}
    \item Depth of network, how many layers the network needs
    \item Bredth of layers, how many nodes each layer has
    \item Size of embedding matrix
    \item Learning rate
\end{itemize}

\subsection{Comparing with a Baseline}
When evaluating the performance of the network one or more baselines are needed as reference points. It is not sufficient to say that the model simply \textit{performs well}, if there is no context \textit{well} could mean just about anything. It is therefore important to have one or more baselines to serve as references. How well the result of the project compares against existing solutions/baselines will determine whether artificial neural networks is a good choice for solving the problem or not. This comparison fulfils the projects purpose of evaluating the performance of ANN for the described problem.
\\\\
If these three objectives have been fulfilled the 

\section{Scope}
This project will evaluate the possibilities of using artificial neural networks to accurately be able to recommend users of a social platform that will be interested in new content posted on the platform.

\subsection{Limitations}
This project will mostly be scientific and will not focus on business applications of the problem.
\\\\
Due  to  technical  limitations  regarding  computing  power  and  available  data  the number of users that will be considered when recommending will be smaller than what would likely be desired in a commercial implementation. This  is  to  make  the training  of  the  network  take  less  time  and  thereby make it possible to perform more experiments. The total number of users that will be considered will be fixed to contain the initial number of users and a buffer for new users. When adding a new user the model remains trained, and only needs to learn from the new user’s data.
\\\\
One issue that arises when dealing with data of this kind is the ethical concern regarding integrity. It is not hard to imagine that a system that could tell whether or not something is of interest to a specific person could be abused. The system could perhaps discover areas of interest that a person would have wished to keep private, this is just one example of potentially many problems. Although this is an important issue the thesis will not cover this subject. 
%Diskutera datan:(notera det här är bara bajs så vi kan diskutera något senare, därför skriver jag på svenska //maxim)
%Vi kunde få ut X antal interaktioner med trådar på reddit, där en interaktion motsvarar en downvote eller upvote.Varje interaktion är kopplad till en specific person i datasettet. Vi får också ut vilken subreddit tråden var skapad på, och eventuell innehåll(content).
\\ %Den här kan typ vara i terminologi, eftersom vi endast förklarar hur datan ser ut.
%I första modellen tog vi bara ut de "positiva" interaktioner, dvs alla med upvotes, och grupperade datan så att varje titel har ett set av användarna (i postgres). För att träna den första modellen var vi bara intresserade av titeln och användarna, dvs titel var vår inputdata och mängden av användarna var vår "target"/"labeled" data, dvs den outputten som vi vill att den ska få. 
%tl;dr input = titel, output = användarna som är intresserade.
\\
%En annan ide är att även använda de negativa interaktioner(downvotes), man kan tolka det på olika sätt: en downvote kan vara att du är inte intresserad av det och då borde man straffa modellen mer om en användare ger en downvote på en titel, det andra sättet och tolka är på är att man är intresserad men håller inte med personen eller tycker inte om sättet någon beskrev det på, men själva ämnet är fortfarande intressant. Det leder till två olika modeller/approaches som man kan undersöka.
\chapter{Theory}
This chapter will discuss some of the important theory behind artificial neural networks and how they will be used to achieve the goal of the project.

\section{Classification}
The most common use for machine learning is classification. It is the problem of given a new data point classify it as one of the k sets that the machine learning algorithm knows. This problem can be solved with most machine learning models. It can be a quite simple problem and often do not require the use of an ANN. %Är det ett verkligen ett "simple problem", beror ju på vad du vill klassificera
\section{Artificial Neural Networks (ANN)}
An Artificial neural network is an attempt to image the workings of a human brain. The network is constructed of nodes which are structured in layers. The nodes take some input and does something with it and feeds it to the next layer, until it gets to the final layer and outputs something interesting, usually a vector of probabilities mapped to the known classes. Between the nodes there are weights, these are the only none constants in the network and are what will change in order to make the network learn.

[IMAGE OF ANN]

\section{Training and Optimisation} %Training and optimisation
An ANN is trained by giving to some in data and comparing the out data with the correct answer by applying some error function. This error is what is to be minimised to make the network learn, and can be achieved by applying backpropagation and some optimisation method, e.g. Gradient descent.

\subsection{Gradient descent, tractability}%Tractability? Borde vi ha ADAM istälelt?
Gradient descent is an optimisation method where one calculates the gradient or diffirential of a function and then takes a small step in the direction of the gradient that minimises or maximises the function, this is repeated until one has found a local (or global) minimum/maximum i.e. the gradient is 0. It has been shown that this works well for convex(?) functions[source] since convex functions have the property that every local minimum is also a global minimum . However, an artificial neural network is not a convex function, but much more complicated.[source] Therefore gradient descent should not be a good method to use in this case, but empirical proof shows that it work well. %Vi ska inte ge vår åsikt här, räcker att säga att Det finns emperiska bevis som säger att gradient descent är bra och ha källa.
When talking about artificial neural networks it is the gradient of the error function that is interesting. Calculating the gradient of the error function with regards to the weights of the network will give an indication as to whether to decrease or increase the weights. The network learns what is good and what is bad by changing its weights to minimise the error function.

\subsection{Backpropagation and how it is used}
Backpropagation is a method used when training ANNs. When you feed a vector into the network it will propagate forward until in reaches the last layer and presents an output. The output is then compared to the desired result via an error function to determine how well the network performed on the given input. Each neural will be assigned an error, the error will then propagate backwards through the network and the weights will be updated. How the weights are updated depends on what optimisation method is used.


\section{Error functions} %Tycker det borde ligga under ANN delen?
When working with machine learning there is a need to be able to measure the error of a certain output. Measuring the error can be used to evaluate the performance of a machine learning technique but when working with artificial neural networks it is also an essential part of optimising the network. In the case of artificial neural network the measured error is used to determine how the network should change in order to perform better. There are a lot of common methods used for measuring errors and different error functions are more suited for some problems.

\subsection{Mean Squared Error}
\begin{equation}\label{eq:meansquared}
    E=\frac{1}{2n}\sum_{i=1}^{n} (target_i - output_i)^2
\end{equation}
The mean squared error function is defined as shown in equation \ref{eq:meansquared} where $n$ is amount of inputs that you give to the error function, $output_i$ is an output from the system and $target_i$ is the actual expected output.

\subsection{Cross Entropy}
Cross entropy function is defined by XXXXX. 

\subsection{LogSumExp}
asdasd

\subsection{Downvote and Upvotes} %Borde inte finnas här, tillhör inte teorin. En del av metod eller diskussion. 
The network is currently learning to predict which reddit users are interested in a given title of a reddit post. It learns this by looking at the interactions that users have had with similar posts. If user have upvoted a post that clearly means that the user found the post enjoyable or interesting, but what if the user has downvoted the post or not interacted at all with it? If a user have downvoted a post that could mean one of two things; either the user did not like the content and do not what to see similar posts again or the user find the post interesting but does not agree with the point of view of the poster, in this case the user most likely want to see more similar posts. Similarly with no interaction, a user might simply not have seen the post not liked it or not cared enough to upvote it. This is a problem of social studies, how do user use their downvote?

\section{Activation functions}
The activation function controls how much of the output from a node is ON. For example if using signum(x) as our activation function then all positive outputs are ON and all negative are off. In order to achieve better results one has to allow more sophisticated outputs than simple ON/OFF-modes. By using non-linear functions you can squash/scale the output. One of the requirements for the network to learn well is to have an activation function that is differentiable in many points.

\subsection{Logistic function}
The logistic function is a non-linear function with a sigmoid curve defined by \ref{eq:sigmoid} that scales the output between $[0, 1]$. 
\begin{equation}\label{eq:sigmoid}
    f(x)=\frac{1}{1+exp(-x)}
\end{equation}
\subsection{ReLU}
ReLU function is a non-linear function defined by $f(x) = max(0,x)$. This function has the benefit of being unbounded as opposed to sigmoid for example whose domain is the interval $[0, 1]$.  
\subsection{Softmax}
Softmax function is defined by $f(x_j) = \frac{e^{\vec{x}_j}}{\sum_{i=1}^{n} e^{\vec{x}_i}} $ where $\vec{x}$ is a vector of $n$ outputs. Softmax scales the vector entries to be between 0 and 1, while making them sum to 1 because of the normalisation.

\section{Recurrent Neural Networks (RNN)}
A recurrent neural network is network that takes time into account. It accounts for time in the sense that the current output is dependent on the previous. This time dependency is often depicted as an ANN that outputs to itself, in reality this is not very useful since it makes the methods for learning (backpropagation) useless. Instead of using a recursive unit, recurrent nets are often modelled as one unit outputing to the next unit in the layer, this unit takes some new input and the output from the previous unit. This processes is called unfolding.%recursive unit? Wut?
\subsection{Long Short-Term Memory (LSTM)}
Skriv om vad en LSTM är

\subsection{Gated Recurrent Unit (GRU)}
Skriv eventuellt om GRUs

\chapter{Method}
This chapter will describe the work process of this project. It will describe what has been done and discuss the decisions taken.

\section{Deciding on a dataset}\label{sec:deciding_dataset}
As described in the initial problem description, deciding on a dataset is essential for the project to proceed. A first step was thus to find a dataset which satisfies the constraints presented in section \ref{sec:select_dataset}. This was needed as a first step since an artificial neural network needs a good source of data to learn from. Since using ANN is a supervised learning technique the data used to train it must be labelled. It thus saves a lot of time if the chosen dataset is already labelled when fetching it or that it can be automatically labelled by some automated process. This is since we were looking for a large dataset to train the ANN. Manually labelling several thousands of data points was never an option when deciding on the dataset. It would be too time consuming for the scope of the project.
\\\\
After searching for difference sources of data it finally came down to two datasets; Ubuntu Dialogue Corpus \parencite{lowe2015ubuntu} and Reddit data (see appendix \ref{appendix:reddit}). The Ubuntu Dialogue Corpus contained forum posts with back and forth discussions whereas the Reddit dataset contained a post and whether a certain user had up or down voted that post. Using The Reddit API (\url{https://www.reddit.com/dev/api/}) it was also possible to extract more information from a post, such as which category it belonged to.
\\\\
The reason for having the Ubuntu Dialogue Corpus and the Reddit up-votes and down-votes data as final considerations comes down to a few factors. Most important was the overall characteristics of the data; it was text content written by users for other users. There were also interaction with the content which could be used to indicate interest in a topic. The availability of the data, its format and size were also factors.
\\\\
After comparing the two datasets the Reddit dataset was chosen. It had a very simple format but primarily it had a very strong indication of user interest compared to the Ubuntu Dialogue Corpus. In the Reddit dataset content was directly linked to users on the platform which had either up- or down-voted it. On Reddit, up or down voting content means that the user clicks a button indicating if they like or dislike a certain post. We made an assumption that an active decision to either up- or down-vote some content would indicate interest and that a general user is less likely to engage in a full discussion (as in the Ubuntu Dialogue Corpus) compared to just clicking a button.
\\\\
An example of what an entry in the Reddit dataset looks like after post-processing is shown in table \ref{table:example_reddit_data}. The raw dataset only contained IDs to posts and categories/subreddits which had to be converted to actual content using the Reddit API. Details on the actual gathering and post-processing of the data is described in section \ref{sec:gathering_data}.
\begin{table}[h!]
    \centering
    \begin{tabu}to 1\textwidth{ X[c] X[c] X[c] X[c] X[c] } 
        \hline
        \textbf{Title} & \textbf{Content} & \textbf{Subreddit} & \textbf{Upvotes} & \textbf{Downvotes} \\
        \hline
        \hline
        &&&& \\
        A Graph of NP-complete Problems & \textit{<empty>} & compsci & izzycat, cypherx, HattoriHanzo & \textit{<empty>}\\
        &&&& \\
        \hline
    \end{tabu}
    \caption{An example data point in the post-processed Reddit dataset showing information about a post (with no content) and which users showed interest in it.}
    \label{table:example_reddit_data}
\end{table}
\\
A question that arose was how to interpret the downvotes or lack of interaction. If a user has upvoted a post we assume that  means that the user found the post enjoyable or interesting, but what if the user has downvoted the post or not interacted at all with it? If a user has downvoted a post that could potentially mean one of two things; either the user did not like the content and does not want to see similar posts again or the user find the post interesting but does not agree with the point of view of the author. In the case where a downvote still represents interest we believe that the user most likely want to see more similar posts. Similarly with no interaction, a user might simply not have either seen the post, has not liked it or has not cared enough to vote on it. Given these combinations of possibilities we decided to make an assumption that if a user had downvoted a post that still indicated interest of the post and if a user had not interacted at all with a post it would mean nothing.

\section{Gathering data}\label{sec:gathering_data}
The Reddit dataset is not so useful in its raw form. As briefly described in section \ref{sec:deciding_dataset} the raw data only consisted of ID references. As an example, a data point in the raw dataset could look like in table \ref{table:raw_reddit_data}. This is far from how the dataset looked like after processing, shown in table \ref{table:example_reddit_data}.
\begin{table}[h!]
    \centering
    \begin{tabular}{ c c c } 
        \hline
        \textbf{Username} & \textbf{Post ID} & \textbf{Vote} \\
        \hline
        \hline
        2bornot2b & t3\_89ko9 &-1\\
        \hline
    \end{tabular}
    \caption{An example data point in the raw Reddit dataset showing that the user 2bornot2b has downvoted the post with ID t3\_89ko9.}
    \label{table:raw_reddit_data}
\end{table}
\\
In the raw form shown in table \ref{table:raw_reddit_data} the data is modelled as a relation from a specific user to a specific post. A post is also represented as a single ID instead of by its title, content and subreddit/category. To get from the raw format to the post-processed format two steps were made. Firstly the ID of a post was used to retrieve more meaningful content from the Reddit API. This included getting a posts title, subreddit and eventual content (a lot of posts only had a title with no other content). Secondly the relations of the data were re-arranged to instead model a relationship from a unique post to several users. This meant that a post was only represented once in the dataset and that the single data point for each post contained all available information about it.
\\\\
To turn a single post ID into meaningful content that could be used as training data in the project the Reddit API was used. As a post ID is only a reference in Reddit's database the API could be used to extract more information by passing the ID to it. The API specifically had an endpoint that could take one or more post IDs as input and return a list with all information available about that post. This enabled the extraction of the post's title and content as well as which subreddit it belongs to. As the dataset is large it would have been ineffective to make the extraction from the post ID manually, so a small so called \textit{scraper} was programmed to automate the task. The Reddit scraper is available at \url{https://github.com/kandidat-highlights/reddit-scraper}.
\\\\
Once more details about each post was extracted the data still had to be re-arranged. The problem with the original format is that each data point models a single user's either up- or down-vote for a post. For the problem described in this thesis the input to the system should be a post and the output should be all users who are interested in it and this is not something that can be modelled with the original format. A format where each post is represented by a single data point was desired. The same data point would also contain all users who has up or down voted the post. The way the final format, as shown in table \ref{table:example_reddit_data}, was achieved by loading all the data into a relational database. By having the data in a relational database it was easy throughout the course of the project to extract different samples of data suitable for some experiments. As an example it could be used to split the dataset into training, validation and testing subsets.

\section{Modelling the artificial neural network}
The general characteristics of the problem is that given some post, one or more users should be recommended based on their previous interest in other posts. A Reddit post in the given dataset is described by three properties; a title, some content text and a subreddit (category). This means that the ANN model should be able to use natural language as input. This property of the problem motivates the use of recurrent neural network to fully capture the input of natural language. A rough initial model was thus to have an artificial neural network with an RNN for the input (first) layer, no hidden layers and a output layer.

\subsection{A basic starting point}
For the output (last) layer each user had to be represented in a way that made it possible to compare them against each other in terms of how likely they were to be interested in a given input post. This in practice meant that the output layer should scale the output of the ANN between $(0,1)$ as this allows for interpretation as probabilities. Two activation functions which does this were considered; the softmax function (see section \ref{sec:softmax_function}) and the sigmoid (see section \ref{sec:sigmoid_function}). As the problem is to recommend one \textit{or more} users it can be seen as a multi-class classification problem, which meant that using the sigmoid activation function would likely perform better. However, as a rough starting point none of the activation functions were ruled out at that stage.
\\\\
An initial model like the one just described was implemented using the machine learning framework \textit{TensorFlow}. For the recurrent neural network, LSTM units were used. At first the network was also modelled to only use the post title as input since this was easier as a first step than using all the features of a post. The input layer was modelled to let one LSTM unit take one word, from the input sentence, at a time as input. The number of LSTM units was chosen by analysing the data in order to capture large amount of titles. However, it is still possible to choose any positive number of LSTM units - hence this is a hyperparameter. The words were transformed to one-hot vectors as described in section \ref{sec:rnn_nlp}. This simple initial model is visualised in figure \ref{fig:first_simple_model}.
\begin{figure}[h]
    \centering
    \includegraphics[width=0.75\textwidth]{figure/ann/ann}
    \caption{A graph visualisation of the first iteration model}
    \label{fig:first_simple_model}
\end{figure}
\\
A problem that quickly revealed itself was the size of the dataset. As the dataset is large, training of the network was very slow even in this simple form. A natural next step was thus to scale down the dataset to allow faster iteration of models. Instead of having a dataset with several thousands of users the dataset was downscaled into only having 5 users and the posts that they had interacted with. By doing the downscaling it was faster to train the network, which in turn meant that it was faster to see if it performed good or not. An assumption made using this strategy is that a model performing well on the downscaled dataset will continue to perform well as the dataset is scaled up again. 
\\\\
With this simple ANN model the goal was to establish some starting point to assure that the basic concept was working. To test that the network was actually learning anything at all we looked for overfittning (see section \ref{sec:overfitting}). If the model could not overfit on the training data using the downscaled dataset it would indicate some conceptual error or misinterpretation of the problem (källa kanske?). However, once we had verified that the model did overfit, and thus was able to learn from the data, the work moved on to prepare a more advanced model for the actual experimentation.
\\\\
At this stage both the softmax and the sigmoid activation functions were used to test the model. With the softmax as the activation function users were picked based on a hyperparameter $k$, where $k$ was the number of users that would get predicted. This meant that the top $k$ users with the highest probabilities would always get picked as predictions from the model. In practice this meant that even if only a single user should have been predicted, the model would always predict $k$ users even if the $k-1$ remaining users shouldn't be predicted at all. The reason for using this method to pick users is based on how the softmax function normalises all probabilities which means that it isn't possible to have a condition to pick all users with a probability higher than a certain constant. This was however supported by the sigmoid activation function, which in the end resulted in the softmax function being removed from further testing. The limit of when to pick a user or not given a probability introduced a new hyperparameter based on how to set the limit. One option considered was to have constant limit $x$, e.g. picking all users with a probability above $x=30\%$. Choosing $x$ then becomes a problem of optimisation. Another option considered was to pick all users with an above average probability. The hyperparameter then becomes to choose which of the two options to use and to choose a value of $x$ if the first option is used.
\\\\
A benefit of using the sigmoid function over the softmax function was that more precise performance measures could be made. It is common to use F1 score when evaluating machine learning models \parencite{yang1999re}, but this was not possible with the softmax function as a constant number of users would always get picked - this meant that the recall would always equal the precision. With the more dynamic way to predict users as the sigmoid function allowed, the recall could be calculated in a more meaningful way which in turn allowed to calculate the F1 score of the model. With the sigmoid as the output function precision, recall and F1 score was measured and used to evaluate the performance of the model. In order to visualise the performance a TensorFlow tool called \textit{TensorBoard} was used.
\subsection{Enhancing the model}
After the performance measures were implemented to calculate precision, recall and F1 score it was possible to accurately compare models against each other. Because of this, the model could now be enhanced and the changes could be evaluated. An enhancement that was easy to implement was regularisation: Both L2 regularisation and Dropout (see section \ref{sec:regularisation}) was added to the model. With regularisation two new hyperparameters were introduced; the dropout probability, $P_{dropout}$, and the L2 factor, $\beta_{L2}$. The options to use either L2 or Dropout regularisation (or both) also became hyperparameters in themselves.
\\\\
In order to add more degrees of freedom to the ANN, the model was enhanced to support hidden layers. The model was implemented so that any non-negative number of hidden layers could be used. For the layers' activation function the ReLU function was used. By adding hidden layers to the model two new hyperparameters were introduced; the number of hidden layers and how many neurons the layers should have. We simplified the model so that all hidden layers have the same number of neurons. This was to keep the number of hyperparameters, and thus the complexity, down to make experimentation easier. Each layer is connected as a chain in a fully connected way, so there are no parallel hidden layers and all nodes between the two consequent are connected. This means that the output from the RNN goes to the first hidden layer and the output from that layer then goes to the second layer and so on - until the output of the last hidden layers goes into the output layer. This is illustrated in a simplified way in figure \ref{fig:ann_hidden_layers}.
\begin{figure}[h]
    \centering
    \includegraphics[width=0.75\textwidth]{figure/method/ann_hidden_layers}
    \caption{A graph visualisation of an ANN with two fully connected hidden layers.}
    \label{fig:ann_hidden_layers}
\end{figure}
\\\\
Further enhancement was to implement pre-trained word embeddings. The word embeddings would be pre-trained with GloVe (see section X.X.X teori). Reason for using word embeddings is that they better capture the relation between words in a language/corpus, GloVe has also been shown to perform better compared to other methods for creating word embeddings such as word2vec \parencite{pennington2014glove}. First step is to evaluate the performance of the already pre-trained embeddings that can be found on (https://nlp.stanford.edu/projects/glove/), the two specific embeddings that were evaluated are pre-trained on "Wikipedia 2014 + Gigaword 5"- and "Twitter"-datasets. During the evaluation different sizes of the embeddings would be tested to find the one that performs best on our task. Second and the last step of word embeddings is to pre-train a word embedding based on the Reddit dataset that was chosen. 
The reason for creating new embeddings is that some words can be community- or platform-specific meaning that there is a chance of capturing more relations between the words. Different dimensions of the embedding matrices will be evaluated against each other. 
\\\\
So far only the title of a post is used as input to the model. Since there is more data available about each post, such as its content and which subreddit it belongs to, the model was also enhanced to incorporate this. For the subreddits, the actual meaning of the subreddit (i.e. the actual text, e.g. \textit{politics} or \textit{humor}) was not of interest. Instead a post's subreddit was encoded as a one hot vector. This vector was fed through a linear layer to control the dimensions and concatenated directly with the output from the existing RNN. The option to either use this extra input or not turned into another hyperparameter.
\\\\
Early in the project we tested to classify subreddits instead of users. The network showed good results for predicting which subreddit the post had been posted in. This led us to believe that it could be useful to first train network at predicting subreddits and then use the weights from this network as initial weights when training on users. The reasoning behind this decision was that a user often like posts that are in the same subreddit. If the network is then already trained to distinguish between posts in a broad term (i.e. in terms of which subreddit they belong to) that could potentially be a good starting point for distinguishing between users' interest. Combining this strategy with using both the title and subreddit of a post as input as previously mentioned was also evaluated.
\\\\
Regarding the content of a post, some analysis on the dataset revealed that only about $x\%$ of all posts actually had any content - the rest only had a title. This lead to the decision to not create a new RNN input layer for parsing the content of a post. Instead the dataset was processed to only show whether a post had any content or not, in a binary way. This was also concatenated with the output of the existing RNN and the subreddit input. Just like for the subreddits; the option to either use the extra input or not turned into a hyperparameter.
\subsection{Tuning hyperparameters}
With the implementation of the model being complete, experimentation and tuning of the introduced hyperparameters was the next step. Hyperparameters are both actual parameters introduced as well as whether to use a certain feature of the model or not. For example, a model with L2 regularisation turned off is different from one with it turned on - it is also not certain that the model will perform better with it turned on. All of the hyperparameters introduced are described below\\
\begin{center}
\begin{tabular}{| c | p{7cm} |}
    \hline
    Hyperparamter  &  description \\ \hline
    Learning rate & A real valued number specifying how heavy the weight adjustments should be during training  \\ \hline
    Batch Size & Positive Integer specifying the size of each batch \\ \hline
    RNN units & Positive Integer specifying the number of units in the RNN  \\ \hline
    Embedding Size & todo \\ \hline
    Pre-trained embedding matrix & todo \\ \hline
    Trainable embedding matrix & A binary choice to either allow or disallow the embedding matrix to be updated during training \\ \hline
    Hidden layers & todo \\ \hline
    Neurons in hiddden layers & Integer specifying how many neurons a particular layer has \\ \hline
    L2 regularisation & todo \\ \hline
    L2 Factor & todo \\ \hline
    Dropout regularisation& Regularisation technique where some neuron, sometimes are deactivated \\ \hline
    Dropout probability & The probability that any given neuron will be turned off \\ \hline
    Use constant prediction limit & todo \\ \hline
    Constant prediction limit & A threshold such that it if it surpassed arecommendation is issued  \\ \hline
    Use subreddit input & A binary choice to either use or not use the given posts subreddit as additional input \\ \hline
\end{tabular}
\end{center}\todo{Fattar inte hur man får in nyrad här :P}
\\
As there are a lot of different combinations of hyperparameters a dynamic and configuration based model was implemented. With this implementation the whole model can be defined using a single configuration file. The configuration file specifies all of the hyperparameters and when the model is run it dynamically builds a computation graph to match them. This setup also enabled scheduling the training and logging the progress of different models in an automated way which made experimentation easier.
\\\\
The goal of the tuning of the hyperparameters was to maximise the F1 score. Under normal circumstances an optimiser would likely be used to find the optimal parameters that maximised the F1 score, but since the model is in fact an artificial neural network the time it takes for training the model makes an optimiser impractical. Instead the optimisation relies on systematic testing of different combinations of hyperparameters in an informed way.
\\\\
The typical workflow when optimising the model is thus to look at previous model that had performed well, identify a hyperparameter that should be changed, test the network with the updated hyperparameter and lastly compare the results. Which hyperparameter that should be changed could be chosen on a few different bases. For example, with the boolean hyperparameters (i.e. use dropout or not) it is easy to make sure both combinations are tested and if one hasn't been tested it is a good hyperparameter to pick.
\subsection{Scaling up}
After an optimal model was found for the downscaled dataset it was time to scale up again....... 
\section{Comparing against a baseline}
When deciding how well a model performs it is compared against a baseline. The baseline puts the accuracy of the model into some context. You might have an accuracy of 90\%, is that good or bad? You need one or more baselines to decide this. 
\subsection{Random classifier}
A random classifier is a model that given an input $x$ selects one of $n$ output values uniformly at random. This results in an expected accuracy of $\frac{1}{n}$. This is a baseline that any real life model needs to beat with confidence.

\subsection{N-grams based model as a baseline}
In this project a model based on n-grams have been used as a baseline \parencite{cavnar1994n}. This N-gram based model first compute  all 1, 2, and 3 -grams in the corpus and for each N-gram $g$, give it a unique number $g_i$ between $0$ and $k$ where $k$ is the total number grams in the corpus. A so called category vector is then built for each user. A category vector for user $u$, $\vec{V_u}$ is a $k$-dimensional vector with entry $\vec{V_{ui}} = n \iff $ the gram $g_i$ occurs $n$ times in the corpus for user $u$. The corpus for user $u$ is defined as all titles labelled with user $u$. Once all the category vectors have been computed the system can start making predictions on how users should be recommended given a new title. Given a new title it is first translated into a title vector $\vec{V'}$ in a similar fashion as for the category vector, the only difference is that the corpus is limited to the title itself. Now $\vec{V'}$ is compared against all the different category vectors $\vec{V}$ using two different metrics, cosine similarity \parencite{steinbach2000comparison} and euclidean distance. The goal with using both cosine similarity and euclidean distance as metrics is that they measure different things. Cosine similarity is based on orientation while euclidean distance is based on magnitude without knowing in advance which one is better suited to our task.  \todo{lägga till fler om man testar fler}
\section{Integration with an application}


%Saker som man är också intressanta: Kolla upp variansen, debugg, hur vi hitta felet etc. 
\chapter{Results}

\section{Comparing to our baselines}
\section{Real world usability}
Skriva om hur det gick overall, jämför hur det fungerade med/utan downvotes. vilka hyperparametrar fungerar bar osv.
\chapter{Discussion}

Here we discuss our amazing results from the method, a lot of tables and stuff. Yay


\section{Performance Measure}%How we calculate precision/validation top k. Consequences etc
We are choosing top 1 user for now. A problem with this blablabla \todo{write more}


\section{Filter Bubble}
An upcoming issue with recommender systems in general is filter bubble.
Filter bubble is a consequence when training the networks on user data. The problem is that users will only be recommended similar information/items to what they have previously liked/encountered, meaning that the users will stay inside their bubble. \\
One of the simple countermeasures against this is if the users themselves are aware of this and actually browse information outside of their comfort zone. This additional information about users will hopefully be taken into account when training the network. \\

\todo{How we treat upvotes/downvotes as interesting, this is in method already. Discuss problems here}
% REFERENCES / BIBLIOGRAPHY
\cleardoublepage
\addcontentsline{toc}{chapter}{Bibliography}
\input{include/back/references}

% APPENDICES
\cleardoublepage
\appendix
\setcounter{page}{1}
\pagenumbering{Roman}			% Capitalized roman numbering starting from I (one)

\chapter{The Reddit dataset}\label{appendix:reddit}
The Reddit dataset was retrieved from \url{https://www.reddit.com/r/redditdev/comments/bubhl/csv_dump_of_reddit_voting_data/}. The dataset was published by the user \textit{ketralnis} who is an employed administrator on Reddit. The data comes directly from the official Reddit databases.
\\\\
The raw data as retrieved contained three fields per data point which represent a user, a post ID and a vote. Every data point in the dataset indicates that a specified user has either up- or down-voted a certain post on Reddit. An excerpt from the dataset looks like shown in Table \ref{table:reddit_raw_excerpt}.
\begin{table}[h!]
    \centering
    \begin{tabular}{ c c c } 
        \hline
        \textbf{Username} & \textbf{Post ID} & \textbf{Vote} \\
        \hline
        \multicolumn{3}{c}{\vdots} \\
        2bornot2b & t3\_899vr & 1 \\
        2bornot2b & t3\_89as2 & -1 \\
        2bornot2b & t3\_89az7 & 1 \\
        2bornot2b & t3\_89b84 & 1 \\
        2bornot2b & t3\_89c4k & 1 \\
        2bornot2b & t3\_89de8 & -1 \\
        2bornot2b & t3\_89e1e & 1 \\
        \multicolumn{3}{c}{\vdots} \\
    \end{tabular}
    \caption{An excerpt from the raw dataset with the field username, post ID and vote.}
    \label{table:reddit_raw_excerpt}
\end{table}
\\
This raw dataset contains $7,405,561$ votes for a total of $31,927$ users and $ 2,046,401$ posts. The dataset only contains votes from users who have actively agreed to make their votes public. The dataset contains up to the latest $1000$ up- and down-votes (for a maximum of $2000$ votes) for each user. The action to either upvote or downvote a post on Reddit is used on the website to rank posts against eachother where an upvote indicates that a user likes a post and a downvote indicates the opposite. 
\begin{figure}[h]
    \centering
    \includegraphics[width=0.75\textwidth]{figure/reddit/reddit_voting}
    \caption{An example of a post on Reddit and how it can be either up- or down-voted. Screenshot from \url{http://reddit.com/} on April 4, 2017.}
    \label{fig:example_reddit_post}
\end{figure}
\\\\
The posts on Reddit are written by users on the platform. A post always has a title, and it always belongs to a so called \textit{subreddit}. The post seen in Figure \ref{fig:example_reddit_post} has the title \textit{Colorized by me: Abraham Lincoln, June 3, 1860} and it belongs to the subreddit \textit{pics}. A subreddit can be seen as a category -- there are many subreddits on Reddit and posts belonging to the same subreddit are usually similar.





\end{document}