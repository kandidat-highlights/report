\chapter{Conclusion}\label{chap:conclusion}
The model proposed in this thesis performs comparably to already existing solutions such as Facebook's fastText and N-grams based model. Because of the computational complexity of using an artificial neural network and the long training time it does not seem like a good approach to the problem. There are some potential solutions that could make the model better as proposed in Chapter \ref{chap:future_work}, but in its current state the model is not so good in real world usage. In the real world scenario of serving recommendations on a large social network the economical cost of running such a complex model would likely be too great -- especially as the complexity increases as more users are introduced.
\\\\
Considering the low precision, even on the smallest dataset tested, it is also unlikely that the model proposed would solve the problem of increased stress described in Chapter \ref{chap:intro} -- rather it would probably contribute to the problem in its current state.
\\\\
We also learned that the dataset used might not be optimal for solving the problem of recommending users in the way examined in this thesis. It is not clear though if it is a problem with this particular dataset or if there is an underlying problem with datasets of this kind (i.e. social platform user behaviours).
\\\\
We can conclude that the best model proposed in this thesis has no real advantages over other similar methods, but there is room for improvement (like described in Chapter \ref{chap:future_work}).