
\makeglossaries
\newglossaryentry{latex}
{
    name=latex,
    description={Is a mark up language specially suited 
    for scientific documents}
}
 
\newglossaryentry{maths}
{
    name=mathematics,
    description={Mathematics is what mathematicians do}
}

\begin{comment} % Känns som acronyms == dictionary, borde välja en av dem? Alla vetenskapliga saker borde vara i teorin såsom ANN.

% acronym är ju mer förkortningar så som ANN, dictionary är ju ordlista förklarar vad ett artfical neural network är.
Artificial neural network -> Mathematical model inspired by biological neural networks (brains)
Learning rate -> parameter that adjust how big of a step is taken towards goal
Baseline -> a criteria for comparison
Deep learning -> a form of an artificial neural network with more than three layers.
Labeled data -> data with known classifications data with known classifications.
Hyperparameters -> parameters manually set to define the structure and rigidity of a neural network
paramater -> A constant number in the model that is updated during training
Softmax  -> function that scales the vector entries to be between 0 and 1, while making them sum to 1
Recurrent neural network -> a form of neural network whose graph can have cycles
Training data -> data used for training the network 
Mentioning and highlighting -> a way of notifying a user.
Tensorflow --> Machine learning library
Overfitting -> Overfitting has occured when the validation error goes up and the traning error goes down. If Overfitting has occured generality has been lost and the model has learnt the training data to well.
\end{comment}